\chapter{Implementation}
\section{Dataset\label{sec:dataset}}
The study of concept drift in malware detection requires a dataset in which samples
can be ordered by some time dimension. This project uses the KronoDroid dataset, introduced in
\cite{guerra2021kronodroid} and retrieved from \cite{guerra2021github}.
The KronoDroid dataset contains 41,382 \textbf{real}
Android malware samples belonging to 240 distinct malware families.
Note that the KronoDroid dataset contains both emulated and real samples; this project
focuses on the real samples only. The samples in KronoDroid are considered
hybrid as they contain 200 static features
(permissions, intents, hardware/software requirements) and
289 dynamic features (system calls).
More importantly, each sample contains four different timestamps.
In this project, we use the \textbf{HighestModDate} timestamp.

\subsection{Malware Families}
Even though there are 240 malware families in the KronoDroid dataset, we focus our attention
on five malware families with the largest number of samples. By doing so, we can
study concept drift and evaluate model performance for a meaningful period of time.
This approach is similar to the one followed by Mishra, A. et al. in \cite{mishra2024cluster}.
The five malware families are:
\begin{itemize}
    \item \textbf{Airpush/StopSMS}: adware that displays unwanted ads and may silently collect
    and forward user data \cite{fsecure2025airpush}. 
    \item \textbf{SMSReg}: riskware discovered as the ``Battery Improve'' application \cite{fsecure2025smsreg}.
    \item \textbf{Malap}: spyware that collects sensitive information from the device \cite{guerra2021kronodroid}.
    \item \textbf{Boxer}: trojan that pretends to be a legitime installer but actually sends premium-rate
    SMS messages unbeknown to the user \cite{fsecure2025boxer}.
    \item \textbf{Agent}: trojan that downloads and installs adware or malware into a victim's
    device \cite{fsecure2025agent}.
\end{itemize}

\subsection{Data Preprocessing}
Each malware sample contains 489 features, 200 static and 289 dynamic. However, some features
are not numerical but rather text.
As a preprocessing step, we remove all non-numerical features using the Polars library
type inference capabilities \cite{polarsdatatypes}.

Additionally, we discard samples whose \textbf{HighestModDate} timestamp seems
to be incorrect. In concrete, only samples with a timestamp
$t$ in the format ``M/D/YYYY'' and year in range $[2000, 2025]$ are kept.
Figure \ref{fig:kronodroid_summary} shows a summary of the KronoDroid dataset
after preprocessing.

\begin{figure}[h!]
    \begin{center}
        \textbf{Kronodroid Dataset (dimensionality = 470)}
        \vspace{0.5cm}
        
        \begin{tabular}{lccc}
            \toprule
            \textbf{Family} & \textbf{Samples} & \textbf{Earliest Date} & \textbf{Latest Date} \\
            \midrule
            Agent               & 2826 & 2008-02-28 & 2020-07-17 \\
            Airpush/StopSMS     & 7760 & 2008-02-29 & 2018-06-15 \\
            Boxer               & 3597 & 2005-01-01 & 2018-06-15 \\
            Malap               & 4018 & 2008-02-29 & 2020-11-11 \\
            SMSreg              & 4989 & 2008-02-29 & 2020-11-09 \\
            \bottomrule
            Total             & 23190 \\
        \end{tabular}
        \caption{Summary of the Kronodroid dataset after preprocessing.}
        \label{fig:kronodroid_summary}
    \end{center}
\end{figure}

Contrary to usual machine learning tasks where we perform data normalization or standardization
at the preprocessing step, concept drift analysis requires a different approach. Details about
how we handle data standardization are explained in Section \ref{sec:workflow}.

\section{Experiment Setup}

\section{Developer Setup}

\section{Experimentation Workflow\label{sec:workflow}}
