\chapter{Implementation}
\section{Dataset\label{sec:dataset}}
The study of concept drift in malware detection requires a dataset in which samples
can be ordered by some time dimension. This project uses the KronoDroid dataset, introduced in
\cite{guerra2021kronodroid} and retrieved from \cite{guerra2021github}.
The KronoDroid dataset contains 41,382 \textbf{real}
Android malware samples belonging to 240 distinct malware families.
Note that the KronoDroid dataset contains both emulated and real samples; this project
focuses on the real samples only. The samples in KronoDroid are considered
hybrid as they contain 200 static features
(permissions, intents, hardware/software requirements) and
289 dynamic features (system calls).
More importantly, each sample contains four different timestamps.
In this project, we use the \textbf{HighestModDate} timestamp.

\subsection{Malware Families}
Even though there are 240 malware families in the KronoDroid dataset, we focus our attention
on five malware families with the largest number of samples. By doing so, we can
study concept drift and evaluate model performance for a meaningful period of time.
This approach is similar to the one followed by Mishra, A. et al. in \cite{mishra2024cluster}.
The five malware families are:
\begin{itemize}
    \item \textbf{Airpush/StopSMS}: adware that displays unwanted ads and may silently collect
    and forward user data \cite{fsecure2025airpush}. 
    \item \textbf{SMSReg}: riskware discovered as the ``Battery Improve'' application \cite{fsecure2025smsreg}.
    \item \textbf{Malap}: spyware that collects sensitive information from the device \cite{guerra2021kronodroid}.
    \item \textbf{Boxer}: trojan that pretends to be a legitime installer but actually sends premium-rate
    SMS messages unbeknown to the user \cite{fsecure2025boxer}.
    \item \textbf{Agent}: trojan that downloads and installs adware or malware into a victim's
    device \cite{fsecure2025agent}.
\end{itemize}

\subsection{Data Preprocessing}
Each malware sample contains 489 features, 200 static and 289 dynamic. However, some features
are not numerical but rather text.
As a preprocessing step, we remove all non-numerical features using the Polars library
type inference capabilities \cite{polarsdatatypes}.

Additionally, we discard samples whose \textbf{HighestModDate} timestamp seems
to be incorrect. In concrete, only samples with a timestamp
$t$ in the format ``M/D/YYYY'' and year in range $[2000, 2025]$ are kept.
Figure \ref{fig:kronodroid_summary} shows a summary of the KronoDroid dataset
after preprocessing.

\begin{figure}[h!]
    \begin{center}
        \textbf{Kronodroid Dataset (dimensionality = 470)}
        \vspace{0.5cm}
        
        \begin{tabular}{lccc}
            \toprule
            \textbf{Family} & \textbf{Samples} & \textbf{Earliest Date} & \textbf{Latest Date} \\
            \midrule
            Agent               & 2826 & 2008-02-28 & 2020-07-17 \\
            Airpush/StopSMS     & 7760 & 2008-02-29 & 2018-06-15 \\
            Boxer               & 3597 & 2005-01-01 & 2018-06-15 \\
            Malap               & 4018 & 2008-02-29 & 2020-11-11 \\
            SMSreg              & 4989 & 2008-02-29 & 2020-11-09 \\
            \bottomrule
            Total             & 23190 \\
        \end{tabular}
        \caption{Summary of the Kronodroid dataset after preprocessing.}
        \label{fig:kronodroid_summary}
    \end{center}
\end{figure}

Contrary to usual machine learning tasks where we perform data normalization or standardization
at the preprocessing step, concept drift analysis requires a different approach. Details about
how we handle data standardization are explained in Section \ref{sec:workflow}.

\section{Experiment Setup\label{sec:setup}}
In this project, we conduct three experiments to evaluate the benefits of using
drift detection for malware detection: static training, periodic retraining,
and drift-aware retraining. In each experiment, we train four machine learning classifiers:
Multilayer Perceptron (MLP), Support Vector Machine (SVM), Random Forest (RF),
and eXtreme Gradient Boosting (XGBoost).

The expectation is that periodic retraining will yield the highest accuracy
for all models accross all malware families while the static training accuracy
will be the lowest. The drift-aware retraining accuracy is expected to be
between the static training and periodic retraining accuracies providing
a good trade-off between accuracy and training efficiency.

\subsection{Definitions}

Let $R = \{R_{O}, R_{K}\}$ be a drift detector model where $R_{O}$ is a
One-Class Support Vector Machine (OCSVM) and $R_{K}$ is a Minibatch K-Means (MBKM).

Furthermore, let $M = \{M_{MLP}, M_{SVM}, M_{RF}, M_{XGB}\}$ be
a machine learning model where $M_{MLP}$ is a Multilayer Perceptron,
$M_{SVM}$ is a Support Vector Machine, $M_{RF}$ is a Random Forest,
and $M_{XGB}$ is an eXtreme Gradient Boosting model.

Let $F$ be a malware family and $D$ be the kronodroid dataset. Moreover,
let $F_{drift}$ be the malware family where we want to detect drift and
$F_{control}$ be a different malware family with respect to $F_{drift}$.
During our binary classification experiments, we will consider samples from $F_{drift}$
to be considered of class $1$ and samples from $F_{control}$ to be considered
of class $0$.

For a given malware family $F$, \textbf{we assume the samples to be ordered}
by the \textbf{HighestModDate} timestamp.
We denote the samples in $F$ as
$F = \{f^{0}, f^{1}, \ldots, f^{N-1}\}$, where $f^{i}$ is the $i$-th sample
and $N$ is the number of samples in $F$. Given the ordered property,
we know that $f^{i} \prec f^{j}$ if $i < j$.

For a given malware family $F$,
let $B$ be a temporal batch of samples. In our experiments, we define batches
of size $b = 50$ samples for a total of $K$ batches. We denote the $i$-th batch as
$$B_{i} = \{f^{i \cdot b}, f^{i \cdot b + 1}, \ldots, f^{(i+1) \cdot b - 1}\}
~\forall~ i \in \{0, 1, \ldots,  K\}, K = \lfloor \frac{N}{b} \rfloor$$.
For example, the first batch is $B_{0} = \{f^{0}, f^{1}, \ldots, f^{49}\}$.
The second batch is $B_{1} = \{f^{50}, f^{51}, \ldots, f^{99}\}$.
The last batch is $B_{K-1} = \{f^{(K-1) \cdot b}, f^{(K-1) \cdot b + 1}, \ldots, f^{N-1}\}$.

Let $B_{i}^{train}$ be the training subset and $B_{i}^{test}$ be the testing subset such that
$$B_{i} = B_{i}^{train} \cup B_{i}^{test}$$
where $B_{i}^{train}$ contains the first $p$ samples in $B_{i}$ and
$B_{i}^{test}$ contains the remaining $b - p$ samples in $B_{i}$. We use
$p = 30$ samples for training and $b - p = 50 - 30 = 20$ samples for testing.
For example, for the first batch $B_{0}$, we have
$B_{0}^{train} = \{f^{0}, f^{1}, \ldots, f^{29}\}$
and
$B_{0}^{test} = \{f^{30}, f^{31}, \ldots, f^{49}\}$.

Finally, let $\mathcal{B}_i = \{B_i, B_{i+1}\}$.

\subsection{Training a model}
For a given pair of malware families $(F_{drift}, F_{control})$ with $K$ batches
defined over $F_{drift}$, we let $X_{train}$ and $X_{test}$ be
subsets of $F_{control}$ such that
$$X_{train} \cap X_{test} = \varnothing$$

Let $D_{i}^{train} = (B_{i}^{train} \cup X_{train})$ and
$D_{i}^{test} = (B_{i}^{test} \cup X_{test})$. Then, a given model
$M_i$ is trained on $D_{i}^{train}$ and evaluated on $D_{i}^{test}$.
Ideally, $|X_{train}| \approx p$ and $|X_{test}| \approx b - p$ to avoid
dataset imbalance. As a result, in all experiments we use
$$X_{train} = \{f_{control}^{0}, f_{control}^{1}, \ldots, f_{control}^{p-1}\}$$
$$X_{test} = \{f_{control}^{p}, f_{control}^{p+1}, \ldots, f_{control}^{b-1}\}$$
Therefore, $|X_{train}| = p = 30$ and $|X_{test}| = b - p = 20$.

The notation $M_i^*$ indicates that the model is an \textbf{optimal model}
that maximizes the accuracy over $D_{i}^{test}$.

\subsection{Static Training\label{sec:static_training}}
In this experiment, we train a model $M_0^*$. We define the overall accuracy $A$
of the static training experiment
as
$$A = \sum_{j=0}^{K-1} \frac{c_j}{|D_j^{test}|}$$
where $c_j$ is the number of correct predictions made by $M_0^*$ over $D_j^{test}$.

Note that this experiment simulates the scenario where we train once
and we do not retrain the model for inferences in the future.

\subsection{Periodic Retraining\label{sec:periodic_retraining}}
In this experiment, we train $M_i^* ~\forall~i \in \{0, 1, \ldots, K-1\}$.

Then, the overall accuracy $A$ of the periodic retraining experiment
is defined as $$A = \sum_{i=0}^{K-1} \frac{c_i}{|D_i^{test}|}$$
where $c_i$ is the number of correct predictions made by $M_i^*$ over $D_i^{test}$.

The periodic retraining experiment simulates the scenario where we
retrain the model every $b$ samples.
\subsection{Drift-Aware Retraining\label{sec:drift_aware_retraining}}
Assume a drift detector $R$ deems the $i$-th batch $B_i$ to be
\textbf{drifting} as a function of only $F_{drift}$.
We define $i$ as a \textbf{drift point}.
Furthermore, let $i < j < z$ be different drift points detected.

In this experiment, we train $M_0^*$ on $D_0^{train}$ and
we calculate a partial accuracy $a_0$ as
$$a_0 = \sum_{l=0}^{l=i-1} \frac{c_l}{|D_l^{test}|}$$
where $c_l$ is the number of correct predictions made by $M_0^*$ over $D_l^{test}$.

Then, we train $M_i^*$ on $D_i^{train}$ and we calculate a partial accuracy
$$a_i = \sum_{l=i}^{l=j-1} \frac{c_l}{|D_l^{test}|}$$
where $c_l$ is the number of correct predictions made by $M_i^*$ over $D_l^{test}$.
Then we train $M_j^*$ on $D_j^{train}$ and we calculate a partial accuracy
$$a_j = \sum_{l=j}^{l=z-1} \frac{c_l}{|D_l^{test}|}$$
where $c_l$ is the number of correct predictions made by $M_j^*$ over $D_l^{test}$.

We repeat this process until we are able to score
$D_{K-1}^{test}$.

Then, the overall accuracy $A$ of the drift-aware retraining experiment
is the sum of all partial accuracies. Note that the number of partial
accuracies is dependent on the number of drift points.

\section{Developer Setup}

\section{Experimentation Workflow\label{sec:workflow}}
