\chapter{Introduction}

Concept drift is the change of the statistical properties of data over time.
Given that machine learning models learn from data, concept drift
can degrade the performance of machine learning models during inference.

Concept drift is a problem in malware detection as
malware is constantly evolving. According to Stamp, M. \cite{stamp2023malware}
attackers modify malware for two purposes: add or modify functionality
and evade detection. As a result, developers and researchers using
machine learning models for malware detection must take concept
drift into consideration when deploying these models into real-life systems.
One single malware program deemed as benign can wreak havoc into a system
by bringing down machines or holding data hostage for ransom. In fact,
the number of cyberattacks, such as a ransomware attacks, made possible
by malware has increased in recent years \cite{bertia2022ransomware}.

One solution to mitigate the effects of concept drift is to constantly retrain
the models used for detection. However, the resource demands of this process
can vary significantly based on the model architecture, 
requiring substantial time, computational power, hardware, and energy.
In an era where significant research efforts focus on \textbf{minimizing} the
\textit{inference} resource consumption of complex machine learning models,
such as Deep Neural Networks (DNNs) and Large-Language Models (LLMs) \cite{energyTrendsInAI},
the addition of constant \textit{retraining} appears inefficient and resource-intensive.

This project describes a framework of \textbf{selective retraining} using
Minibatch K-Means (MBKM) and One-Class Support Vector Machines (OCVSM)
for drift detection. We conduct three experiment scenarios
to illustrate the benefits of using drift-detection: static training,
periodic retraining, and drift-aware retraining. In each experiment,
we train four machine learning classifiers: Multilayer Perceptron (MLP),
Support Vector Machine (SVM), Random Forest (RF), and eXtreme Gradient Boosting (XGBoost)
using the Kronodroid Dataset \cite{guerra2021kronodroid}.

This project report consists of five sections. The first section provides background information
about MBKM, OCSVM, the Kronodroid dataset, and the ML models used.
The second section explains the experiment setup. The third section details the implementation.
The fourth section presents experiments, results, and analysis.
The final section concludes the project.